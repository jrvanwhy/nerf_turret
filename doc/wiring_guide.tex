\nonstopmode
\documentclass{article}

\linespread{1.5}

\title{Wiring Guide}
\author{}
\date{}

\setcounter{secnumdepth}{0}

\begin{document}
	\maketitle

	\section{RCX}
	The motor which pulls the ``accelerator'' (spins up the flywheels) should be plugged into port A.
	The motor which pulls the trigger should be connected to port C.
	Both sets of connections should face ``downwards'' (or ``backwards'') relative to the RCX/Nerf gun, such that the floss is pulled over the top of the winch reels
	when the motors activate (in other words, the winches rotate ``backwards'' relative to the nerf gun).
	These should use the two extremely long cables.

	The light sensor should be connected to port 1.

	\section{NXT}
	The elevation control motor should be wired to port A via a long cable.
	The left pan motor should be connected to port B and the right pan motor should be connected to port C, both using short cables.
	In total, when the turret is pointing ``forward'', the motors (from left to right) should be connected to ports A, B, and C, respectively.

	The touch sensor for elevation calibration should be connected to sensor port 1 using a long cable.

	The light sensor should be connected to sensor port 4 using a short cable.
	This has a reason: a light sensor connected to ports 1-3 will flash when the NXT is powered on, which will send a signal to the RCX program to activate the winch motors.
	Port 4 does not have this issue.

\end{document}
